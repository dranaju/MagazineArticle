\section*{Deep Reinforcement Learning}

The goal in deep reinforcement learning is to control an agent attempting to maximize a reward function. 
The deep Q-network (DQN) algorithm \cite{mnih2013playing} was capable of human level performance on many Atari video games by estimating the actions of an agent.
However, while DQN could solve problems on complex observation spaces, it only can handle discrete action spaces. 
It is noticeable that many tasks, on the robotic control, have continuous action spaces. 
So DQN cannot be applied to continuous domains and it is necessary to use another algorithm that can handle this type of problems.

\subsection*{Deep Deterministic Policy Gradients}

The deep deterministic policy gradients (DDPG) algorithm consists of an actor-critic method that uses approximation functions that can learn continuous action space policies. The algorithm makes use of a neural network for the actor network and other for the critic network. These two networks computes the action prediction  for the current state and generates a temporal-difference error signal for each time step. The input of the actor network is the current state, and the output is a real value representing an action chosen for a continuous action space. The output of the critic is simply the estimated Q-value of the current state and the action given by the actor.

The biggest challenge of learning in continuous action space is exploration.
To confront this challenge is needed to construct a exploration policy $\mu'$ by adding a noise sampled from a noise process $\mathcal{N}$ to the actor network policy, defined as:
\begin{equation}
\mu' = \mu(s_t) + \mathcal{N}
\end{equation}
where $\mathcal{N}$ can be chosen in a way to suit the environment.
Being the Ornstein-Uhlenbeck process \cite{uhlenbeck1930theory} the most used to generate temporally correlated exploration efficiency in physical control problems.

In general, to train and evaluate a policy function coming out of the actor network and a value function coming out of the critic network, with thousands of simulated trajectories temporally correlated, leads to the introduction of huge amounts of variance in the approximation of a true Q-function.
It is suggested to use a replay memory to store the experiences of the agent during training \cite{schaul2015prioritized}.
This means, saving the states, actions, rewards and new states that the agent explored during the episode.
And then, randomly sampling experiences to use  for learning in order to break the temporal correlations within different training episodes.

The replay of experiences allows the intelligent agent to learn from recent memories, increasing  the learning speed and breaking undesirable temporal correlations.
Even with a short memory it is possible to see a substantial improvement in the agent performance.
Despite the application of a replay memory slows the agent learning, a better performance is achieved.

Without counting the replay memory used, it is needed to make a target network to generate targets for the temporal-difference error that can regulate the learning and improve stability.
The target network contains a copy of the actor and critic, however, with ``soft" updates.
This means that the target network do not copy directly the weights of the actor and critic network.
The weights, represented by $\theta$, of the target network are then updated by:
\begin{equation}
    \theta' \leftarrow \tau\theta + (1-\tau)\theta
\end{equation}
with $\tau \ll 1$. 
Where the target is represented by the apostrophe.
So the target values of the network change slowly and allows an improvement in the stability of the learning.
